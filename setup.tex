% use this file to include packages and create macros

\usepackage[utf8]{inputenc}

\usepackage{subcaption}
\usepackage{framed}
\usepackage{amsfonts,amsmath}
%\usepackage{amsfonts,amssymb,amsmath,amsthm}

\usepackage{listings}
\lstset{basicstyle=\ttfamily, numbers=left, numbersep=5pt, numberstyle=\tiny, stepnumber=1, escapechar=\!, columns=fullflexible, morekeywords={true, false, a, @prefix}}

\usepackage{booktabs} % For formal tables

\usepackage{tikz, tabularx} % for drawing
\usetikzlibrary{patterns}
\usepackage{pgfplots}
\usetikzlibrary{positioning, arrows}  % use: above left of, etc
% tikz macros
\newcounter{c}
\newcounter{head}
\newcounter{tail}

\newcommand{\Symbol}[4]{
    \node at (#1,#2) [draw, circle] (#3) {#4};
}

\newcommand{\Nodeset}[5]{
    \setcounter{c}{0}
    \foreach \v in {#4} {
        \node (#1-\thec) [draw] at (#2, -{\thec/2+#3}) {$\v$};
        \setcounter{tail}{\thec}
        \stepcounter{c}
    }
    
    \ifnum \thec=0%
        \node (#1-\thec) [draw, dashed, minimum width=.45cm] at (#2, -{\thec/2+#3}) {};
        \setcounter{tail}{\thec}
        \stepcounter{c}
    \fi
    
    \stepcounter{c}
    \setcounter{head}{\thec}
    
    \foreach \v in {#5} {
        \node (#1-\thec) [draw] at (#2, -{\thec/2+#3}) {$\v$};
        \stepcounter{c}
    }

    \ifnum \thec=\thehead
        \node (#1-\thehead) [draw, dashed] at (#2, -{\thec/2+#3}) {};
    \fi
    
    \draw[->, >=stealth] (#1-\thehead) to (#1-\thetail);
}

\usepackage{hyperref} % for links
\usepackage{amsmath}

\usepackage[ruled, vlined, linesnumbered]{algorithm2e} % For algorithms
\renewcommand{\algorithmcfname}{ALGORITHM}

% math symbols
\newcommand{\ldb}{\mathopen{\lbrack\!\lbrack}}
\newcommand{\rdb}{\mathclose{\rbrack\!\rbrack}}
\newcommand{\derives}{\Rightarrow^*}
\newcommand{\T}{\mathit{traces}}
\def\bigO{\hbox{$\mathcal {O}$}}
%\newcommand{\tuple}[1]{( #1 )}
\newcommand{\set}[1]{\{ #1 \} }
\newcommand{\pr}[2]{#1 \rightarrow #2}
\newcommand{\Rule}[2]{#1 \to #2}
\newcommand{\R}[2]{\ensuremath{R(\mathit{#1},\mathit{#2}})}
\newcommand{\rules}{P}
\newcommand{\nonterminals}{N}
\newcommand{\terminals}{\ensuremath{\Sigma}}
\newcommand{\observers}{O}
\newcommand{\nil}{\textbf{nil}}
\newcommand{\ttable}{T_G}
\newcommand{\res}{\textit{res}}
\newcommand{\resm}{\res^\bullet}
\newcommand{\resu}{\res^\circ}
\newcommand{\vis}{\textit{vis}}
\newcommand{\vism}{\vis^\bullet}
\newcommand{\visu}{\vis^\circ}
\newcommand{\sta}{\textit{dscr\/}}
\newcommand{\stam}{\sta^\bullet}
\newcommand{\stau}{\sta^\circ}
\newcommand{\m}{^\bullet}
\renewcommand{\u}{^\circ}

% SPARQL commands
\newcommand{\SELECT}{\texttt{SELECT}}
\newcommand{\WHERE}{\texttt{WHERE}}
\newcommand{\AND}{\texttt{.}} % the 'AND' in SPARQL is actually a dot '.' -- ciro
\newcommand{\OPT}{\texttt{OPT}}
\newcommand{\OR}{\texttt{OR}}
\newcommand{\USING}{\texttt{USING}}
\newcommand{\FROM}{\texttt{FROM}}
\newcommand{\FILTER}{\texttt{FILTER}}
\newcommand{\var}[1]{\texttt{?#1}}
\newcommand{\inv}[1]{\overline{#1}} % 
%\newcommand{\inv}[1]{\char`\^\!#1} % SPARQL's inverse operator is '^' and it comes before the predicate

%\newcommand{\rep}[3]{#1 \ensuremath{\sim\!\! >} #2 \ensuremath{<\!\!\sim} #3}

\newcommand{\rep}[3]{\texttt{<}#1\texttt{>}#2\texttt{<}#3\texttt{>}}

% For drawing RDF graphs
\newcommand{\rdfedge}[4]{\draw [->, sloped, >=stealth, above, #4] (#1) to node {\texttt{#2}} (#3);}
\newcommand{\rdfnode}[3]{\node[draw, #3] (#1)
{\texttt{#2}};}
\newcommand{\rdfproperty}[5]{\node[minimum width=0, minimum height=0, #4 of #1] (#1-#2)
{\texttt{#3}};\rdfedge{#1}{#2}{#1-#2}{#5};}
\newcommand{\rdfloop}[3]{\path (#1) edge [loop #2,->, >=stealth, thick,red] node {#2}  (#1);}


%\newtheorem{theorem}{Theorem}
%\providecommand\definitionautorefname{Theorem}

% \newtheorem{lemma}{Lemma}
% \providecommand\definitionautorefname{Lemma}

% \newtheorem{corollary}{Corollary}
% \providecommand\definitionautorefname{Corollary}

% \newtheorem{definition}{Definition}
% \providecommand\definitionautorefname{Definition}

\newtheorem{grammar}{Grammar}
\providecommand\grammarautorefname{Grammar}

\usepackage{newfloat} % For query environment
\DeclareFloatingEnvironment[fileext=frm,placement={!ht},name=Query]{query}
% \captionsetup[query]{labelfont=bf}

\newcounter{numberInTrivlist}

\newenvironment{numtrivlist}{\begin{list}
          {\rm \arabic{numberInTrivlist})}
          {\usecounter{numberInTrivlist}
          \setlength{\leftmargin}{0pt}
          \setlength{\rightmargin}{0pt}
          \setlength{\itemindent}{12pt}
          \setlength{\listparindent}{0pt}}}
                            {\end{list}}

\newenvironment{itemizedTrivlist}{\begin{list}
          {\rm ~\hspace{2mm} $\bullet$\ } 
          {\setlength{\leftmargin}{0pt}
          \setlength{\rightmargin}{0pt}
          \setlength{\itemindent}{12pt}
          \setlength{\listparindent}{0pt}}}
                            {\end{list}}

\newcounter{inlineenum}
% \renewcommand{\theinlineenum}{\alph{inlineenum}}
\newenvironment{inlineenum}
  {\unskip\ignorespaces\setcounter{inlineenum}{0}%
   \renewcommand{\item}{\refstepcounter{inlineenum}{(\theinlineenum)~}}}
  {\ignorespacesafterend}


%article macros: rules to translation and reduction
\newcommand{\sem}[1]{[ \! [ #1 ] \! ]}
\newcommand{\guard}[1]{[ #1 ]}
\newcommand{\tuple}[1]{\langle #1 \rangle}
\newcommand{\trad}[1]{{\cal T}\sem{ #1 }}
\newcommand{\setof}[1]{\{ #1 \} }
\newcommand{\eval}[2]{\mathit{Eval}_{#1}(#2)}
\def\vertices{{\cal V}}
%\def\tr{\rhd}    % transition relation
\def\tr{\rightarrow}    % transition relation
\newcommand{\fpfrule}[2]{{{{\displaystyle #1}\over{\displaystyle #2}}}}
\newcommand{\pfrule}[2]{\begin{array}{c} #1 \\ \hline #2 \end{array}}
\newcommand{\pfruleN}[3]{\begin{array}[b]{c} #1 \\ \hline #3 \end{array} \ (#2)}
%my macros
%\newcommand{\set}[1]{\lbrace #1 \rbrace}

\newcommand{\rec}{\texttt{rec}}




% comments
\usepackage{ulem}
\normalem
\newcommand{\ourremark}[3]{\noindent{{\footnotesize\sffamily\textcolor{#2}{#3\textsuperscript{#1}}}}}
\newcommand{\mm}[1]{\ourremark{\tiny Martin}{red}{#1}}
\newcommand{\ciro}[1]{\ourremark{\tiny Ciro}{blue}{#1}}
\newcommand{\gsv}[1]{\ourremark{\tiny Semyon}{orange}{#1}}
\newcommand{\umberto}[1]{\ourremark{\tiny Umberto}{purple}{#1}}

%\hypersetup{draft}

\pgfplotsset{compat=1.14}
